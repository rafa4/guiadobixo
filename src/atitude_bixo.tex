\begin{secao}{Atitude, bixes!}

Na USP, os alunos têm a liberdade e apoio de se organizarem
para montar grupos de debates, ciclos de palestras, grupos
de desenvolvimento e até mesmo grupos para jogarem alguma coisa (como
RPG, Magic, Yu-Gi-Oh ou algum esporte).
Portanto, caso vocês tenham algum projeto em mente, não hesitem
em se organizarem com seus amigos e se informarem sobre como avançar com essa
ideia. Lembre-se de que seus veteranes estão aí para te aconselhar e tirar
eventuais dúvidas.

É possível também juntar-se com alguns amigos e formar grupos de
estudo, seja para alguma matéria com a qual vocês tenham dificuldade, para
discutir aquele EP/lista de exercícios que ninguém está conseguindo
fazer ou simplesmente estudar algum tópico de interesse mútuo.

A seguir, temos (majoritariamente) alguns dos exemplos dos grupos que foram 
direta ou indiretamente criados por alunos do nosso Instituto!

% RDs --------------------------------------------------------------------------
\begin{subsecao}{RDs}

Antes de mais nada, RD significa Representante Discente. O RD é um aluno que
representa nossos interesses frente aos diversos conselhos e comissões
existentes, sendo um forte elo de ligação entre professores e alunos. O RD
ajuda a tomar decisões que impactam todo o IME, como autorização para festas,
mudanças no currículo, aumento de vagas na FUVEST, quantidade de bolsas,
reformas, mudança no corpo docente (às vezes lutamos para tirar algum
professor), enfim, coisas desse tipo e muitas mais.

Acho que já deu para perceber o quanto é importante ter um estudante em cada um
desses conselhos. Infelizmente, não costumamos preencher todas as vagas. Isso se
deve ao desinteresse de alguns ou falta de tempo da maioria de seus veteranes.
E vocês são quem tem mais tempo para fazer as coisas funcionarem aqui, já que
ainda não sabem o que é rec, DP, trabalho, estágio etc. Portanto, se quiserem
fazer alguma coisa pelo lugar onde vocês, bixes, vão estudar, está aí
uma dica.

Como RDs vocês poderão entender melhor o funcionamento do Instituto e ajudar no
processo de melhorá-lo. Vocês também poderão melhorar o relacionamento entre
estudantes, professores e servidores e entender como os professores pensam.
As eleições são organizadas pela direção do instituto no final do segundo
semestre e o mandato é de um ano.

Aqui vai um breve resumo do que mais ou menos acontece em cada um dos colegiados
nos quais temos direito a representante(s):

No IME, temos 24 cargos de RD com 34 vagas no total, sendo 17 reservadas
para graduação, 14 para pós e 3 livres. Todos têm direito a um suplente.
Caso o RD não possa ir a alguma reunião ou, por algum motivo da vida, tenha
que abandonar o cargo, o suplente assume em seu lugar e o cargo não fica
sem representante.

Existem diferentes níveis de hierarquia na administração.

{\bf As CoCs,
Comissões Coordenadoras de Curso (Lic, Pura, Estatística, Aplicada, BMAC e
Computação)} são as mais próximas dos alunos. Temos um cargo de aluno em cada
comissão. São comissões pequenas, que tratam dos problemas internos de cada
curso: mudança de currículo, requerimentos, optativas etc. São subordinadas
à CG e ao conselho do relativo departamento. Analogamente, temos um cargo em cada
Comissão Coordenadora de Programa (de Pós).

{\bf Os Conselhos de Departamento (MAT, MAE, MAC e MAP)} têm uma dinâmica um
pouco diferente das CoCs: são mais formais. Cada conselho se reúne (quase)
mensalmente e são formados (em geral) por mais pessoas, sendo que existem
regras sobre participação dos diferentes níveis hierárquicos de
professores (Titular, Associado, Doutor e Assistente). Nesses conselhos, além
de aprovar algumas das decisões das Comissões Coordenadoras de Curso e de
Programa (pós) e distribuição de carga didática, são discutidos reoferecimento
de curso, revisão de prova, supervisão das atividades dos docentes,
afastamentos (temporários ou não), contratação de professores e muitas outras
coisas.
Os Conselhos de Departamento são subordinados à Congregação e ao CTA.

{\bf A Comissão de Graduação (CG)}, basicamente, avalia requerimentos,
mudança/criação de cursos e jubilamentos. Analogamente, existe a Comissão de
Pós-Graduação (CPG). Ambas são subordinadas à Congregação.

{\bf A Comissão de Cultura e Extensão (CCEx)} quase nunca tem reunião. Cuida
das atividades de extensão: Matemateca, CAEM etc.

Também há comissões mais específicas, como a comissão de estágio, a comissão de
pesquisa (do doutorado) e o Centro de Competência em Software Livre (CCSL), da
computação.

Os dois conselhos mais importantes são o CTA e a Congregação, ambos presididos
pelo Diretor.

{\bf O Conselho Técnico e Administrativo (CTA)} cuida de todas as questões não
acadêmicas: Orçamento, reformas, avaliação dos funcionários, Xerox etc. É
formado pelos quatro chefes de departamento, diretor, vice-diretor, um
representante dos funcionários e um RD.

{\bf A Congregação} é o órgão máximo do Instituto. Inclui muitos professores, a
maioria titular. São 3 RDs de graduação e 2 de Pós. Basicamente,
nesse órgão, são rediscutidas e aprovadas (ou não) muitas das decisões
dos órgãos subordinados. Os membros da Congregação têm voto na eleição para
Reitor e Vice-Reitor.

Bom, caso não tenha ficado claro desde o começo desse texto, percebam que é
muito importante ter um aluno em cada um desses conselhos. Se estiverem tendo
problemas com professores, requerimentos etc., ou simplesmente quiserem saber
o que anda acontecendo, procurem o RD certo pra conversar. Perguntem,
participem, votem e façam o IME um lugar melhor.

Sobre a eleição dos RDs: A eleição oficial para os RDs acontece no final do ano
(então fiquem atentos!) e é organizada pela diretoria do instituto. Os
interessados devem preencher um formulário de inscrição e levar até a
Assistência Acadêmica do IME, que vai organizar todos os inscritos e abrir um
processo de eleição online (em que todo IMEano pode votar). Quando a eleição
estiver se aproximando, vocês receberão (vários) e-mails com os documentos
necessários, prazos e links de votação.

%REFTIME
O resultado da eleição anterior com os RDs de 2020 pode ser encontrada no site
do IME:
\url{www.ime.usp.br/eleicoes-estatutarias}

\end{subsecao}


% Rede Linux -------------------------------------------------------------------
\input{rede_linux.tex}

% FLUSP ------------------------------------------------------------------------
\input{flusp.tex}

% IME Júnior -------------------------------------------------------------------
\input{ime_jr.tex}

% MaratonUSP -------------------------------------------------------------------
\input{maratonusp.tex}

% USPCodeLab -------------------------------------------------------------------
\input{usp_code_lab.tex}

% USPGameDev: Pesquisa e Desenvolvimentos de Jogos na USP ----------------------
\input{usp_game_dev.tex}

% IMEsec -----------------------------------------------------------------------
\input{imesec.tex}

% Hardware Livre ---------------------------------------------------------------
\input{hardware_livre.tex}

% Tecs: Computação Social ------------------------------------------------------
\input{tecs.tex}

%FIXME GAMBIARRA para não quebrar página num lugar zuado
\pagebreak

% Diversime --------------------------------------------------------------------
\input{diversime.tex}

% Existimos! -------------------------------------------------------------------
\input{existimos.tex}

%FIXME GAMBIARRA para não quebrar página num lugar zuado
\pagebreak

% Comissão de Acolhimento da Mulher! -------------------------------------------
\input{cam.tex}

%FIXME GAMBIARRA para não quebrar página num lugar zuado
\pagebreak

% CinIME -----------------------------------------------------------------------
\input{cinime.tex}

% Grupo A5 ---------------------------------------------------------------------
\begin{subsecao}{Grupo A5}

\figurapequenainline{grupo_A5}

Somos um grupo de estudantes de graduação e pós-graduação do IME e organizamos 
eventos acadêmicos no instituto. Tudo começou em 2012, quando dois alunos da 
pura perceberam que alguns assuntos muito interessantes sobre matemática 
e sobre o meio acadêmico nem sempre eram desenvolvidos em sala de
aula, e resolveram organizar um evento para levar um destes assuntos aos demais
estudantes instituto. Assim, juntamente ao CAMat, organizaram um ciclo de
palestras sobre "Os 7 Problemas do Milênio", e o evento foi um sucesso.  Vários
professores e estudantes começaram a pedir que mais eventos como este fossem
organizados, e foi então que um deles teve a ideia de criar um grupo
independente das demais instituições do IME (sim, o Grupo A5 é independente. Não
é vinculado ao CAMat e nem a nenhum outro grupo, apesar de aceitar parcerias em
alguns eventos), com a finalidade de complementar a formação dos estudantes do
IME e de quem quiser participar, levando palestras e outros eventos sobre temas
relevantes e não explorados no currículo, de forma gratuita e com linguagem de
fácil entendimento.  E assim, no final de 2013, o Grupo A5 oficialmente nasceu,
com nome e logo e com novos integrantes no grupo.  Desde então, não paramos
mais. Em 2014 organizamos o ciclo de palestras "IC ou Não IC? - Eis a Questão",
em 2015 organizamos o evento "História da Matemática", que foi indicado como um
dos destaques de Cultura e Extensão de 2015 pelo IME. E em 2018 realizamos, em 
parceria com o Existimos, o ciclo de palestras "Mulheres no Mundo Corporativo",
que foi um sucesso! 

Para mais informações do Grupo A5 e dos eventos já organizados por nós, acessem
nosso site \url{www.ime.usp.br/~acinco} (ainda está em construção, mas em breve os
vídeos das palestras e demais informações estarão lá). E não deixem de curtir
a página Grupo A5 no facebook: \url{www.fb.com/pagina.GrupoA5} (é aqui que
vocês terão em primeira mão os detalhes de tudo o que for feito por nós).

Vale ressaltar que o Grupo A5 é formado e mantido por estudantes do IME, então o
sucesso e a continuidade do grupo dependem de todos; começando por vocês,
bixes. Então venham, assistam, dêem sugestões, participem. E se gostarem,
juntem-se ao nosso grupo!

\end{subsecao}


% Olimpíadas de Matemática e Informática ---------------------------------------
% Maratona de Programação ------------------------------------------------------
\begin{subsecao}{Olimpíadas de Conhecimento}

\begin{itemize}

\item{\bf Matemática: }

Bom pessoal, se vocês entraram no IME, muito provavelmente já participaram
de alguma Olimpíada de Matemática no Ensino Fundamental e/ou Médio. A
boa notícia é que vocês vão poder continuar participando se quiserem,
e quem nunca participou tem a oportunidade de começar agora.

Mas por que participar? As Olimpíadas Universitárias de Matemática são uma
oportunidade de se divertir resolvendo problemas difíceis de Matemática e agregar
valor ao currículo ao mesmo tempo. Elas são parecidas com as Olimpíadas de
Ensino Médio, mas com conteúdo de Matemática da graduação (essencialmente
Cálculo, Análise, Álgebra Linear, Álgebra, Combinatória e Teoria dos Números),
mas com enfoque em problemas que exigem criatividade e técnicas mais inovadoras,
muitas das quais vocês provavelmente não verão durante toda a graduação.

De quais olimpíadas podemos participar? Como alunos de graduação, vocês podem
participar da Olimpíada Iberoamericana de Matemática Universitária (OIMU),
Olimpíada Brasileira de Matemática (OBM) e Olimpíada Internacional de
Matemática (IMC).

Como fazemos para nos preparar? Os sites institucionais dessas olimpíadas
têm todo o material necessário para vocês que querem estudar e se preparar
para elas.

Como fazemos para participar? Inscrevam-se pelo site ou entrem em contato com
o professor Yoshiharu. Para o IMC aconselha-se ter ganhado medalha na OBM,
já que é necessário apoio financeiro do IME por ser uma olimpíada internacional.

%REFTIME
Mas nós, bixes, temos chance? Como foi o desempenho de IMEanos nelas? Nós
obtivemos sucesso nestas olimpíadas. Ganhamos medalhas em todas as três
competições e o resultado mais recente foi uma medalha de bronze na IMC e ouro
na OBM.

Se tiverem alguma dúvida, não hesitem em perguntar a algum veterane sobre os
Olímpicos!

Links institucionais:

\begin{description}
  \item[] \url{http://oimu.eventos.cimat.mx}
  \item[] \url{http://www.imc-math.org}
  \item[] \url{http://www.obm.org.br}
\end{description}

\item{\bf Informática: }

\textit{``Informática? Vocês mexem com Word, Excel e PowerPoint então?''}

Responder essa pergunta já virou rotina para competidores da Olimpíada
Brasileira de Informática (OBI). Não, Informática não é Word. Oras, então o que é a OBI?

A OBI é uma competição de lógica, matemática e computação. As provas envolvem
alguns problemas que você deve resolver com programas de computador.

Esta competição, na graduação, é exclusiva para ingressantes recém formados do
ensino médio. Quer dizer que vocês são a nossa única esperança de trazer mais
gloriosas medalhas ao IME! Isso também quer dizer que essa é a sua única chance
de participar da OBI, uma competição relativamente tranquila comparada à
Maratona de Programação.

Para participar, basta falar com o MaratonIME
(\url{https://www.ime.usp.br/~maratona/}), um grupo de extensão focado nesse
tipo de competição que promete te ajudar a se inscrever e se preparar, ou com o
Professor Carlinhos (\url{http://www.ime.usp.br/~cef/}).

Para mais informações, acessem \url{http://olimpiada.ic.unicamp.br/}.

\item{\bf Maratona de Programação: }

À primeira vista, a Maratona de Programação pode soar um tanto
surreal. Nerds correndo pela USP ao mesmo tempo que resolvem
problemas de computação e matemática? Infelizmente esse não
é o caso.

A Maratona de Programação se resume à resolução de problemas.
Se você adora resolver desafios, quebrar a cabeça com novos
e excitantes problemas e acumular toneladas de dinheiro, esse
é o lugar perfeito para você!

A competição consiste em uma série de problemas que englobam
temas como programação dinâmica, grafos e estruturas de dados.
Times de três pessoas devem resolver a maior quantidade de
desafios em cinco horas de programação. E tudo isso com direito
a um lanche gratuito durante a prova.

Mas não temam, bixes. Não é só por que vocês acabaram de entrar que
a probabilidade de ganhar uma medalha seja nula. Inclusive, na primeira
fase da maratona, uma equipe de bixes tem vaga garantida para a
fase brasileira.

Além da fama, constantes pedidos por autógrafos e dinheiro de sobra,
a maratona também vai lhes trazer um conhecimento muito mais
adiantado em relação ao dos seus colegas de classe, e até oportunidades
de emprego em empresas de renome, como Google, Facebook e IBM.

Se vocês se interessaram pela maratona e querem saber os horários dos
treinos, como participar ou saber mais, acessem:

\begin{description}
  \item[Site:] \url{http://www.ime.usp.br/~maratona}
  \item[Site da competição nacional:] \url{http://maratona.ime.usp.br/}
\end{description}

\end{itemize}


\end{subsecao}


%FIXME GAMBIARRA para não quebrar página num lugar zuado
\pagebreak

% Fala Sério -------------------------------------------------------------------
\input{fala_serio.tex}

\end{secao}
