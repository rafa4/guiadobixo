\begin{secao}{Dicas}

Como todos os bixes chegam perdidos, aqui vão algumas dicas pra vocês não ficarem perguntando o tempo todo:

%TODO Checar isso aqui.
{\bf Bancos e Caixas Eletrônicos:} agências do Santander, Bradesco,
Banco do Brasil, Caixa e Itaú na Av. Prof. Luciano
Gualberto; caixas eletrônicos perto do bandejão da Física, em frente à reitoria,
no CEPE, enfim, em lugares aleatórios.

{\bf MatrUSP:} é um site criado por alunos do BCC para simular grades horárias
para matrículas de disciplinas. Assim, vocês conseguem saber direitinho se suas
disciplinas vão coincidir e quanto tempo livre vocês vão ter para ficar na
vivência entre as aulas. Acessem: \url{http://bcc.ime.usp.br/matrusp}

{\bf USPAvalia:} outro site criado por alunos do BCC (pois é, olha só!) que
contém inúmeras avaliações da comunidade sobre disciplinas oferecidas e
seus respectivos professores. Ótimo para saber se vale a pena ou não pegar
essa ou aquela turma de uma dada disciplina, e vocês mesmos podem contribuir com
suas próprias avaliações e comentários! Acessem: \url{https://uspavalia.com}

\begin{subsecao}{Cultura na USP}

Vale aqui lembrar que a entrada em vários desses museus é gratuita para alunos da
USP, então vale muito a pena visitar!

{\bf Museu de Arqueologia e Etnologia (MAE):} ao lado da Prefeitura do Campus,
possui um dos maiores acervos de artefatos arqueológicos e etnográficos do Brasil.

{\bf Museu de Arte Contemporânea (MAC):} próximo ao CRUSP, nele são expostas produções
artísticas nacionais e estrangeiras.

{\bf Museu do Brinquedo:} fica na Faculdade de Educação, Bloco B. Seu acervo conta 
com itens datados do início do século XX, incluindo brinquedos, jogos, materiais
pedagógicos e um acervo fotográfico. Tem a exposição "Cenas Infantis", que fica na
bilbioteca da Faculdade de Educação.

{\bf Museu do Crime da Polícia Civil:} na Academia de Polícia perto do P1. Seu acervo
reúne ferramentas, objetos e documentos utilizados em delitos de grande repercussão, 
além da história de criminosos cujos atos ficaram marcados na imprensa e na sociedade
brasileira.

{\bf Museu do Instituto Oceanográfico:} adivinha? Esse museu mantém uma exposição voltada
a dinâmica, a estrutura e a biodiversidade dos oceanos.

{\bf Museu da Geociências:} Lá mesmo. Conta com amostras de rochas, gemas, meteoritos 
e fósseis.

{\bf Instituto Butantan:} Próximo à História. É um dos maiores acervos de pesquisa
biológica do mundo, conta com a presença de diversas cobrinhas.

{\bf Museu de Anatomia Veterinária:} Perto do P3 (portão da Corifeu). Seu acervo possui
uma coleção de dados e fotos de esqueletos, além de modelos anatômicos e animais
preservados.

{\bf Museu Paulista, vulgo Ipiranga:} fora da USP, no Parque da
Independência - S/N  - Ipiranga. Atualmente está fechado para visitação,
pois está sendo restaurado.

{\bf Museu de Zoologia:} também fora da USP, na Av. Nazaré, 481  -
Ipiranga. Sua exposição abriga uma série de animais empalhados, fósseis,
réplica de fósseis etc.

{\bf CinUSP Paulo Emílio:} Dentro do campus, próximo ao bandejão central, existe uma sala de cinema. Durante todo o ano ocorrem várias mostras cinematográficas, nas quais são exibidos inúmeros filmes. As sessões são gratuitas e a programação pode ser conferida no seguinte site: {\tt http://www.usp.br/cinusp/}


\end{subsecao}

\begin{subsecao}{Onde beber?}

Se vocês são dos bixes que curtem entornar os canecos de vez em quando, então
agora devem estar pensando, até que enfim vamos falar de algo que presta!
Lembre-se de levar um veterane para pagar a ele algumas doses, pois é graças a
eles que vocês estão recebendo essas dicas, então sem mais delongas, aqui vão
alguns lugares firmeza para se fazer isso à vontade:

{\bf FAU:} Na vivência da FAU sempre vende cervejas. Para chegar, basta ir no 
primeiro andar à direita até o fim.

{\bf Física:} Embora seja a Física, lá é um lugar gostoso para comer alguns
salgados na lanchonete e tomar algumas cervejas na atlética.

{\bf FFLCH:} Vá até o prédio da História e Geografia e vá até o Aquário, fica 
logo à esquerda na entrada do vão, se estiver alguém lá dentro eles vendem cerveja.

{\bf FEA:} Entrando na FEA, ande até o final, siga reto depois da biblioteca,
passe por um portãozinho, vire à esquerda e tem uma entrada antes do restaurante.
Lá é a vivência da FEA onde vendem as cervejas geladinhas.

%FIXME Qual é a situação atual do QiB?
{\bf ECA:} Famosa Quinta i Breja, adivinha que dia da semana isso acontece?
Acertou, toda quinta-feira na prainha da ECA a partir das 19h!

{\bf Rei das batidas:} Muito famoso não só por quem estuda na USP, o Rei,
como é carinhosamente chamado, fica fora da USP, saindo pela P1. Vende
diversas batidas e, é claro, cerveja...

{\bf Bar do frango:} Não se sabe qual é o verdadeiro nome desse bar, mas ele é
uma alternativa ao Rei, quando este se encontra muito lotado. Apesar do péssimo atendimento e do aspecto horrível do lugar é bom para beber sem
tumulto. É frequentado principalmente no começo do ano. Se encontra atrás do
Rei.

É claro que também há as festas, que ocorrem em qualquer lugar da USP e quase todas as sextas tem, e nelas
há ainda outras misturas alcoólicas impossíveis.

É nosso dever informar também que o álcool é uma substância altamente viciante
e, quando bebida em excesso, pode trazer graves consequências à sua saúde, às
vezes à saúde de outra pessoa, à sua família e principalmente ao seu bolso.
Lembrem-se também que é proibida a venda de bebidas alcoólicas dentro da USP, então
não saiam da vivência dos respectivos lugares com a latinha na mão.

\end{subsecao}
\end{secao}
