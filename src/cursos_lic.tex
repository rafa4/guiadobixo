\begin{subsecao}{Licenciatura}

Olá, bixes! Se vocês chegaram até aqui, então parabéns!

Não só porque passaram na FUVEST ou no ENEM, mas também porque entraram na
Licenciatura em Matemática, mesmo sendo chamados pelos seus colegas de doidos ou
loucos, entre outros adjetivos simpáticos.

Se vocês ainda não sabem exatamente o que farão com o seu curso, tentaremos
explicar, mas esperamos que tenham em mente uma coisa: vocês vão ser
professores, aqueles que sanam as dúvidas dos outros, então foquem em
aprender o suficiente para serem professores maravilhosos (talvez um pouco diferentes
de vários professores que vão encontrar aqui na faculdade...). E como fazer isso? 
Temos algumas sugestões e comentários:

Primeiramente, não caiam na conversa de seus veteranes e colegas dos bacharelados
que insistem em dizer que o curso de licenciatura é mais fácil que o deles. São
cursos diferentes.

Um bacharel é um pesquisador. Portanto, usa a Matemática explorando seus
problemas em aberto na esperança de solucionar algum deles e, consequentemente,
criar outros mais.

Já um licenciado, é um professor. Apto a lecionar na Escola Básica e com
competências para fazer o aluno compreender esse universo tão mágico que é a
Matemática. Se vocês chegaram até aqui com vontade de serem professores, então
podem ter tido bons professores de matemática. Inspirem-se neles, os
superem. Somente através de vocês o mundo poderá ver que a Matemática também é legal. 
Ainda mais aquela aprendida na escola, pois a parte divertida fica para ser 
aprofundada na faculdade, e é o que vocês estarão fazendo nesses "n" anos que se seguirão.

Vocês terão uma base de vários ramos da matemática: Geometria, Cálculos,
Estatística, Álgebra, Computação, entre outros. No decorrer do curso, vão
descobrir em qual área acadêmica preferem fazer as disciplinas de
aprofundamento, onde deverão escolher as matérias em que querem se especializar.
Podem ser tanto na área de Física (para vocês se tornarem professores de Física
também!), quanto nas de Educação, Estatística, Álgebra, Computação, Matemática
Aplicada em Saúde Animal e o que mais a sua imaginação (e o Jupiterweb) permitirem.
Como podem ver, o curso é um "coringa", se comparado com os outros.

Além disso, a formação de vocês também vai abranger questões como: o contexto
social do aluno, a preparação para a sala de aula, a psicologia da educação e
diversas metodologias de ensino. Para isso, vão fazer disciplinas na
Faculdade de Educação, que irá prepará-los melhor nesse contexto (ou, pelo
menos, deveria. É, vão se acostumando, bixes...)

Com a reforma do MEC para as licenciaturas, implantada na USP em 2006,
vocês também farão as ATPAs (Atividades Teórico-Práticas de Aprofundamento).
Seus veteranes provavelmente vão chamá-las carinhosamente de AACCs (Atividades
Acadêmico-Científico-Culturais), que é o nome antigo. Essas atividades são:
projetos de iniciação científica, oficinas e cursos de aperfeiçoamento,
participação em eventos e outras ações que enriqueçam a formação profissional e
pessoal. Fiquem espertos, pois terão que correr atrás de tudo isso sozinhos.
Estejam atentos com essa matéria, são 200 horas para cumprir! Mas, vejam pelo
lado bom: várias dessas atividades são muito prazerosas!

Como podem ver, o curso vai lhes dar um leque bem amplo de escolhas que
podem transformá-los em excelentes professores; basta vocês irem atrás de
se informar e participar das atividades. Portanto, bixes, ajam!

\begin{subsubsecao}{Dicas da cartola!}

Agora algumas dicas tiradas da cartola:

Vocês podem fazer diversas atividades acadêmicas e muitas outras não acadêmicas e
consequentemente mais divertidas, bixes, porém tudo tem um preço.

\begin{enumerate}
\item Podem passar o ano todo só participando de festas e levando o curso nas
  coxas, o que vai ser bem divertido e estenderá o tempo de permanência na
  faculdade para vocês, mas cuidado: tudo tem limite, e jubilar, apesar dessa
  palavra se assemelhar a júbilo, nesse caso não é uma boa coisa!
\item Podem passar o ano todo na biblioteca estudando até rachar, ser o nerd da
  turma (e vê se passa cola, viu?) e diminuir com isso o tempo de faculdade.
  Vocês seriam bons candidatos a RD, já pensaram nisso? Isso gera coisas boas
  com relação a bolsas e empregos, então também vale a pena, mas não vão se
  esquecer de fazer amizades, pois é a única coisa que realmente importa.
\item O tão difícil meio-termo. É um ideal difícil de ser conquistado; afinal,
  quem já viu um nerd em todas as baladas ou o baladeiro de plantão que só tira
  10? Aliás, vão se acostumando, pois o 10 aqui no IME é virtual... Vocês vão
  entender isso, mais cedo ou mais tarde! Bom, se tudo der certo, vocês vão
  tirar boas notas (leia-se algo entre 5 e 7), serem mais conhecidos/chegados
  dos professores por se formar de um a três anos a mais que o normal e ainda
  vão participar das melhores baladas! Se isso não é bom, então vou voltar a
  fazer as minhas listas de Cálculo...
\item Passem em Cálculo; se tem algo que vale a pena dizer é isto: passem em
  Cálculo. Bombar aqui vai atrapalhar muito! Claro que tem outras matérias muito
  importantes para passar também, mas essa é pré-requisito para muitas coisas.
  Façam uma lista de coisas que têm pré-requisito para cursar e deem prioridade
  a elas. 
\item Façam amigos. São eles que vão te ajudar a prosseguir. Muitas vezes
  pensamos em desistir, e os amigos são aqueles que em último caso nos arrastam,
  literalmente, para o caminho certo!

\end{enumerate}

\end{subsubsecao}
\quadrinhos{3}

\end{subsecao}
