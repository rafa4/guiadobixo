\begin{secao}{CCSL - Centro de Competência em Software Livre}

Se vocês já se cansaram da definição tradicional de software (``hardware é a
parte que você chuta, software é a parte que você xinga'') e prefere pensar que
``software é a parte com que você faz o que quiser'', o Software Livre é para
vocês. Baixar de graça? Sim! Copiar para os amigos? Claro! Fuçar, mexer e
hackear? Com certeza! E aqui no IME ele faz parte do ensino e da pesquisa,
literalmente, há décadas! Tanto que o IME tem o CCSL — Centro de Competência em
Software Livre — que existe para dar apoio a essas atividades internamente e
promover a difusão do Software Livre fora da Universidade também.

Vocês podem participar das atividades (palestras, eventos) que o CCSL
promove e também colaborar com elas; ou podem mergulhar em algum dos
projetos apoiados por ele, melhorando o código. Alguns desses projetos
podem vir a ser ótimos temas para seus TCCs ou para trabalhos de
iniciação científica (com bolsa! :D). Então vocês ainda juntam o útil ao
agradável. Visitem http://ccsl.ime.usp.br e inscrevam-se na lista de discussão,
conheçam os projetos e fiquem por dentro das próximas atividades do
Centro!

Ah, mas vocês ainda nem sabem direito o que é Software Livre? Não se preocupem.
Vocês ainda vão ouvir falar muito dele, tanto no IME quanto fora dele. %Além
%disso vai haver uma apresentação a respeito no começo do semestre, fiquem
%ligados! %REFTIME

\end{secao}
